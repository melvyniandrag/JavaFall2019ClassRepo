\documentclass[12pt]{article}
\usepackage[breaklinks=true]{hyperref}
\usepackage[margin=1in]{geometry}

\usepackage{color}

\definecolor{pblue}{rgb}{0.13,0.13,1}
\definecolor{pgreen}{rgb}{0,0.5,0}
\definecolor{pred}{rgb}{0.9,0,0}
\definecolor{pgrey}{rgb}{0.46,0.45,0.48}

\usepackage{listings}
\lstset{language=Java,
  showspaces=false,
  showtabs=false,
  tabsize=2,
  breaklines=true,
  showstringspaces=false,
  breakatwhitespace=true,
  commentstyle=\color{pgreen},
  keywordstyle=\color{pblue},
  stringstyle=\color{pred},
  basicstyle=\ttfamily,
  frame=single,
  moredelim=[il][\textcolor{pgrey}]{$$},
  moredelim=[is][\textcolor{pgrey}]{\%\%}{\%\%}
}

\title{Java Collections}
\author{
	Melvyn Ian Drag
}
\date{\today}


\begin{document}
\maketitle

\begin{abstract}
$java.util$ provides many containers. These containers are widely used in Java programming and are implementations of the great data structures you hear about in data structures \& algorithms classes. In today's lecture we'll have a look at a few of them and consider when we would want to use them.
\end{abstract}

\section{Introduction}
A java collection is a container you can use to store a bunch of values. For example, in your program you may need to store a bunch of ages - in this case you would

\begin{lstlisting}
...
int[] ageArray = {19, 25, 13, 41, 15};
...
\end{lstlisting} 

or you might create a class called 'Person' and when want to store a bunch of People. In this case you might

\begin{lstlisting}
Person[] peopleArr = { new Person(), new Person(), ...};
\end{lstlisting}

in these two examples I am using arrays. While arrays are fine for simple collections of items, there are some distinct disadvantages of using arrays compared to using a more elegant container. There are some cases where using an array is extremely problematic. One major disadvantage is that the array size is fixed. Once you create an array of 5  elements , you cannot add a sixth person to the array.  

There are many Java Collections. For example, 

\begin{enumerate}
\item ArrayList
\item LinkedList
\item Vector
\item PriorityQueue
\item ArrayDeque
\item HashSet
\item LinkedHashSet
\item TreeSet
\item HashMap
\item TreeMap
\item LinkedHashMap
\item ConcurrentListSkipMap
\item WeakHashMap
\item Stack
\item etc.
\end{enumerate}

We are going to focus on just a few of them today. Namely, we will compare

\begin{enumerate}
\item ArrayList
\item LinkedList
\item HashSet
\item TreeSet
\end{enumerate}


\section{The Collections Interface}

\section{What is an ArrayList?}

\section{What is a LinkedList?}

\section{Initializing a List<T> with an ArrayList<T> or LinkedList<T>}
Weird thing that Java programmers do. Java programmers use this pattern to achieve a bunch of complex things that you won't learn about in this class. Grab a good book and learn to write a Java Web App or Android App and you will see this pattern used in certain ways to achieve particular ends.

\begin{lstlisting}
List<String> l = new ArrayList<String}(){
\end{lstlisting}

\section{What Is a HashSet?}

\section{What is a TreeSet?}


\section{Timing Comparison}

\end{document}
