\documentclass[12pt,a4paper]{article}
\include{the_workbook_preamble}
%
\begin{document}
%
%title and author details
\title{\textbf{Week01 - Getting Started in the Cloud}}
%\author[1]{Melvyn Ian Drag}
\date{\today}
%
\maketitle

\begin{abstract}
In this class we discuss the final keyword, the static keyword, access
specifiers, control flow, primitive types vs classes,enum classes, loops.  We
will review pillars of OOP to further cement them in your mind.
\end{abstract}

\section{In no particular order}
This class will be about a variety of topics. The topics are all basic java
topics and are presented in random order. I usually like to build a big lecture
around a central idea, but today we are focusing on lots of small ideas!

\section{The final keyword}
This word is for constants. Sometimes in your programs you want to create a
value and make sure that it doesnt change. This would be a constant.

For example, the following values should never change during your program's
runtime:

\begin{itemize}
\item final double PI = 3.14;
\item final int SECS\_PER\_MIN = 60;
\item final int MINS\_PER\_HOUR = 60;
\item final double LITERS\_PER\_GAL = 3.81; // I think
\item \textit{get more suggestions from students}
\end{itemize}

Also note that it's customary to put your final values in capital letters. Not
required by the java compiler, but it is nice to do this for your colleagues.
Java programmers expect constant values to be written in capital letters.

\section{The static keyword}
For you learning Java, the word "static" is probably confusing. What does it
mean? We won't worry about the history of this word, I'll just tell you what it
means:

Static means "just one". It means there is "just one" of the thing in your whole
program. It often has the implication that the static thing is shared by your
whole program. It also often has the implication that it is not tied to a
particular instance of a class. 


Example 1: The main method is static. Why? it doesnt require you to new up a
class instance, you can call main without a class instance.

Write simple program. then ask students if we've instantiated any objects - did
we call "new"? No. But we're still calling a method if we run this. Which
method? main().

Example 2: Standard example
Every time you create an object, you increment the counter.

Understanding the static keyword will be very important when we talk about
multithreading in a couple of weeks.

\section{control flow}

\subsection{if, elif, else}

\subsection{switch}

\section{loops}
\subsection{while}

\subsection{do while}

\subsection{for}
 
\section{Data types}
\subsection{Almost everything in Java is an Object}
All objects in Java, including arrays, inherit some basic functionality from the
Object class.So when you make a base object in Java, it isn't really the base
object, there is the Object class below it. 

An example of methods you get from the Object class are toString() and clone().

clone() method would be fun to play with - is it a deep copy? Shallow copy? or
just the sae exact object with a new name?

\subsection{Primitive types are not objects.}
\begin{itemize}
\item int
\item double
\item float
\item char
\item byte
\item bool
\end{itemize}

Interesting that in java the byte datatype is 16 bits. This will be very
important in the coming weeks when we look at the way that computers do binary
encoding of text data. In many languages a char is 8 bits! But java has an 8
bit type called "byte" instead. Very interesting history behind why this is.

\subsection{Java classes that had me confused}
When I learned Java I was in a hurry. Had no book and no guidance, so I was
mainly just reading code I found online, compiling it, and tweaking it to see
what happened.

At that time I had no idea what the point of these types were! If there is an
int, why does Java give us Integer? Are they the same? Different?

\begin{itemize}
\item Integer
\item Double
\item Character
\item Boolean
\item Float
\item Byte
\end{itemize}

I can tell you now - these types are the Object form the underlying primitive.
So an Integer is like an int, but it's an object instead of a primitive. 

One immediate benefit of this is that you can use certain libraries that expect
objects. Some libraries dont want primitives.

\end{document}
