\documentclass[12pt]{article}
\usepackage[breaklinks=true]{hyperref}
\usepackage[margin=1in]{geometry}

\usepackage{color}

\definecolor{pblue}{rgb}{0.13,0.13,1}
\definecolor{pgreen}{rgb}{0,0.5,0}
\definecolor{pred}{rgb}{0.9,0,0}
\definecolor{pgrey}{rgb}{0.46,0.45,0.48}

\usepackage{listings}
\lstset{language=Java,
  showspaces=false,
  showtabs=false,
  tabsize=2,
  breaklines=true,
  showstringspaces=false,
  breakatwhitespace=true,
  commentstyle=\color{pgreen},
  keywordstyle=\color{pblue},
  stringstyle=\color{pred},
  basicstyle=\ttfamily,
  frame=single,
  moredelim=[il][\textcolor{pgrey}]{$$},
  moredelim=[is][\textcolor{pgrey}]{\%\%}{\%\%}
}

\title{Java Collections}
\author{
	Melvyn Ian Drag
}
\date{\today}


\begin{document}
\maketitle

\begin{abstract}
In case you are unsure of how to do the midterm exam, we will review some ideas.
We will play around with the ppm image format. Instead of hiding text in the last bit of the red channel, well do the greenscreen thing from last week, but with ppm instead of png. After this exercise, I expect you to be comfortable reading, manipulating, and writing ppm files.
\end{abstract}


\section{About Today's Lecture}
Today's lecture is a lab rather than a lecture. The main point of today's lecture is to have you think about more about image manipulation so that you are able to complete your exam. Your performance in this lecture will be graded as an extra credit assignment.

\section{View PPM}

\section{Create a PPM}

\section{Read it into memory}

\section{Any pixel that is background should be made white}

\section{Write the modified image back out to a file}

\section{View the image and see that it is modified}


\end{document}
